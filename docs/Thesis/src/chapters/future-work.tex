\chapter{Future Work}
\label{chapter:future-work}

At this point the system is not ready for production use.
As seen from the \autoref{chapter:evaluation}, the centralized ledger
is a bottleneck.
It is also a single point of attack --- the system is not fully decentralized
until the ledger is also decentralized.
For this purpose, an integration with a blockchain ledger is seen as the
natural next step.
In preparation for production-ready state and the usage of a blockchain,
the data that is written to the ledger should be minimized.
In particular, the verification results, reputation scores, etc. can all be
stored in the system itself and only have a signed hash of the data written
to the blockchain.

The PoR algorithm we are using is not public, meaning that the verifier requires
a secret key to verify the proof.
Keeping secrets in a decentralized system is not possible.
It must be replaced by a public PoR algorithm.
Most modern PoR algorithm papers propose both a secret-based and public versions.
The PoR algorithm we are using in this thesis also has a public version,
albeit less efficient and more difficult to implement.

On the implementation side, the major bottleneck is the communication between the different modules.
A few of them have to be locked behind a mutex, which makes some parts of the system sequential.
Ideally we want to parallelize all the operations that can be parallelized.
This is achievable by using a pool of workers for the modules that support it --- ledger,
verifier, and grpc server.
