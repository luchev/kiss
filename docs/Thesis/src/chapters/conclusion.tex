\chapter{Conclusion}
\label{chapter:conclusion}

In this thesis, we have presented a system for decentralized storage of data
with a focus on data integrity.
It shows how we can use PoR to bring these guarantees that cloud storage providers
offer to decentralized storage systems, and what are the costs of doing so.
We analyze how effective proof of retrievability is in detecting data corruption
and how it can be used as a part of a decentralized storage system
to increase the trust in the data stored.

Adding PoR to the system allows peers to keep track of what peers are misbehaving,
and thus, the system can be self-healing --- it can detect and remove said peers.
While this solves the integrity problem, it is not a universal solution to attacks.
It is always possible for a powerful enough adversary to take control of the system
by controlling a majority of the peers.
However, in other cases, the system can be resilient to attacks on the data,
and can provide guarantees that the data is stored correctly.
This comes at a cost of time and resources, as the system has to continuously
monitor the peers and the data they store.

Combining PoR, a reputation system, and the modern decentralized networks,
we can achieve a storage system that has similar guarantees as a cloud storage provider
as listed in \autoref{s3-requirements}.
Solving the integrity problem was a crucial step to achieve the requirement for data durability.

If a client has specific requirements, that can only be met by using decentralized storage,
e.g., the client wants the data to not be censored, it is worth using a system like this.
However, if a client needs simple data storage, that is durable and available,
it is recommended to rely on cloud storage providers.
These providers can usually provide better speeds, and they rely on proof of retrievability internally,
which allows them to provide guarantees about the data integrity.
