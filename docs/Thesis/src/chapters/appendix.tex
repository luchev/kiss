\appendix
\chapter{Appendix}
\label{appendix}

\section{Organization of the repository}

The repository can be found at \url{https://github.com/luchev/kiss}.

The project is organized as follows:

\dirtree{%
.1 config \DTcomment{peer configs}.
.1 data \DTcomment{default directory for peer's data}.
.1 docs \DTcomment{documentation}.
.1 logs \DTcomment{log files from peers}.
.1 proto \DTcomment{protobuf definitions}.
.1 src \DTcomment{rust source code}.
.2 build.rs \DTcomment{code generation pre compilation}.
.2 main.rs \DTcomment{main program}.
.1 submodules \DTcomment{external packages}.
.1 justfile \DTcomment{automation scripts}.
}

The whole project is controlled via just scripts (\url{https://github.com/casey/just}).
The justfile contains all the necessary commands for building, running and testing the project.
Some important commands are:
\begin{itemize}
    \item \texttt{just build} - builds the project.
    \item \texttt{just run X} - builds and runs the project with config \texttt{config/X.yaml}.
    \item \texttt{just test} - runs the unit tests.
    \item \texttt{just run-many count} - runs \texttt{count} peers.
    \item \texttt{just thesis} - compiles this document.
    \item \texttt{just clean} - cleans logs, database, kills running peers --- this is run before every benchmark test.
    \item \texttt{just put data} - inserts \texttt{data} into the network as a file.
    \item \texttt{just get uuid} - retrieves the file under \texttt{uuid} from the network.
\end{itemize}

\section{Running the project}

To run the project, you need a unix machine,
the Rust compiler in nightly mode (cargo 1.74.0-nightly),
the \texttt{just} tool, grpcurl, the protobuf compiler, and Docker.
Compiling the latex files is done additionally with \texttt{tectonic}.

After cloning the repository we have to initialize the submodules as seen in \autoref{lst:gh-clone}.
\begin{lstlisting}[language=bash, caption={Cloning and setting up the repository}, label={lst:gh-clone}]
git clone git@github.com:luchev/kiss.git
cd kiss
git submodule update --init --recursive
\end{lstlisting}

For the minimum setup we need two peers running \texttt{just run base} and \texttt{just run peer1}.
Once we have two peers running we can insert a file with \texttt{just put <data>}.

\section{Running the benchmarks}

Running the benchmarks is done by using the just commands and reading the results from the logs.
Most benchmarks output \texttt{info} level logs.
Running the benchmarks is done as seen in \autoref{lst:bench}.
While there is a script for running the benchmarks that uses the same commands, 
it is not recommended because before each benchmark we want to make sure that
the system is in a fully initialized state.
Sometimes the peers need a couple of seconds to connect to each other.
The Immudb docker image might take a while to start as well.
All these factors and more cause the first couple of requests to either fail or take
much longer than the rest, which skews the results.
For this reason it is often needed to rerun the last step \texttt{just put-bytes-times}
to make sure we get consistent results.

\begin{lstlisting}[language=bash, caption={Running the benchmarks}, label={lst:bench}]
just clean # remove logs, data, reset database
just run-many 10 # run 10 peers
just run base &>base.log & # run the base peer
sleep 2 # wait for the peers to initialize and connect
just put-bytes-times 1000000 100
\end{lstlisting}
