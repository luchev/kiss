\chapter{Introduction}
\label{chapter:introduction}

Cloud storage is a key component of modern computing.
It allows users to store and access their data from anywhere in the world.
A lot of the popular cloud storage providers are centralized, meaning owned and operated by a single entity.
On one hand, this is good for the user because the entity provides a friendly interface and customer support
in exchange for a fee.
On the other hand, the user has to trust the entity to keep their data safe and private and not misuse it.
They also might prefer to pay with their storage or computation power instead of money, or 
they might not want to be locked into a single provider.
Such cases are often when the user is a university, an NGO, or occasionally private users.

To address such use cases and the concerns with cloud storage, decentralized storage systems have been proposed.

We will focus on the most popular type of decentralized storage, which is decentralized hash tables (DHT).
They are often used as the basis of NoSQL databases, such as Cassandra \cite{cassandra},
which are used by companies such as Facebook, Twitter, and Netflix, in order to serve millions of users.
While in the example above these databases are used within the premises of a cloud or a private datacenter,
they are mostly used as decentralized storage solutions,
therefore they focus on privacy, security, fault tolerance, and resistance to censorship.

The goal of this thesis is to design and implement a secure and fault-tolerant decentralized storage system
that can compete with centralized storage systems, whose features we've listed in \ref{section:cloud}.
Decentralized networks work on top of untrusted networks (unregulated networks with potential security risks),
such as the Internet.
In particular, we will focus on the integrity of the data stored on the network,
as this is the main problem that is not addressed by existing decentralized storage systems.
In the following sections, we will introduce decentralized networks,
define what security and integrity means, discuss the motivation for this work,
and propose a solution to the integrity problem in modern decentralized networks.
We will then talk about the requirements for the system and how we will evaluate it.

\section{Overview of decentralized networks}

Decentralized storage is implemented on top of peer to peer (P2P) networks.
They are a type of network where each node acts as both a client and a server.
This means that each node can request and provide services to other nodes.
There is also no central server, although there may be a central directory to help nodes find each other.

Decentralized networks can be classified into two categories --- unstructured and structured.
Structured networks have a predefined topology and are more efficient in terms of routing and resource discovery.
Often used topologies are circular and binary trees.
The most popular structured network is Kademlia \cite{kademlia}, which uses a binary tree topology.
It is also the foundation for most modern decentralized storage systems.

We discuss decentralized networks in more detail in \ref{section:distributed-storage}.

\section{Security}

Security is a major concern in decentralized networks because
the network is open to anyone, and there is no central authority to enforce rules,
in contrast to centralized storage systems.
Peers can join and leave the network at any time, and they can lie about their identity and the data they store.
We call such peers malicious.

We go into more detail what security means in \ref{subsection:security}.
In this thesis we will focus on a subclass of attacks - storage attacks.

\subsection{Storage attacks}
\label{section:storage-attacks}

Storage attacks are mostly ignored by the literature, because decentralized networks are often designed to
drop old files based on some criteria, such as popularity or age.
This is done to save space and to keep the network up to date.
However, if we want durable storage, we need to address these attacks.

Storage attacks can be classified into two categories:
\begin{enumerate}
    \item \textbf{Data availability attacks} --- an attacker claims to store data, but does not.
    \item \textbf{Data integrity attacks} --- an attacker claims to store data, but stores different data.
\end{enumerate}

Checking if a node stores the data it claims to store can be as simple as asking the node to return the data.
However, this is a very inefficient and bandwidth-consuming method.
Ideally, we would like to reduce the amount of traffic between nodes and still be able to check
if the data is stored correctly.

\section{Proof of Retrievability}

Proof of retrievability (PoR) refers to the ability of a prover to convince a verifier that it is storing a file.
This could be as simple as sending the whole file, sending a hash of the file, or running a modern
cryptographic PoR protocol.
We are mainly interested in the last one, as it could be the most efficient one.

In this thesis we will explore the use of PoR protocols to solve the
integrity problem and prevent storage attacks.
We will look into how performant PoR protocols are and whether they are viable for decentralized networks.

We will dive deeper into PoR in \ref{section:por}.

\section{Requirements for the system}
\label{section:requirements}

If we want to design a decentralized storage system that is similar and can compete with centralized systems,
we need to provide the same guarantees and features:
\begin{enumerate}
    \item \textbf{Scalability} --- the system should be able to handle many users and data.
    \item \textbf{Performance} --- the system should be fast.
    \item \textbf{Resilience} --- the system should be able to recover from attacks.
    \item \textbf{Reliability} --- the system should be able to recover from failures.
    \item \textbf{Availability} --- data stored on the network should be accessible at all times.
    \item \textbf{Integrity} --- data stored on the network should not be possible to be tampered with.
    \item \textbf{Security} --- data stored on the network, and accessing the data should be secure.
\end{enumerate}

\textbf{Scalability} is covered by most networks, regardless of being centralized or decentralized.
Allowing many peers to join the network is a key feature of decentralized networks.
\textbf{Performance} is also covered by most modern networks, as they provide $O(\log n)$ query time.
\textbf{Resilience} is for the most part covered by the security additions to the networks, such as S/Kademlia.
An exception is the storage attacks, which are not addressed.
\textbf{Reliability} is covered by rebalancing the network when a node leaves or joins.
Most networks also provide some form of redundancy, which allows the network to recover from failures.
\textbf{Availability} follows from reliability and resilience.
\textbf{Security} can be solved by using encryption, cover traffic, onion routing, etc.
\textbf{Integrity} is not the focus of existing networks, as discussed in \ref{section:storage-attacks}.

These requirements have some overlap.
In order to achieve resilience, we need to guarantee the integrity of the data.
In order to achieve performance, we need optimized way to check the integrity of the data.
And to achieve availability, we need the above two.
We discuss how to solve the integrity problem in \autoref{section:solving-the-integrity-problem}

If we want a system to compete with centralized storage systems, we
need to be able to provide guarantees about the data stored on the network.
A lot of the popular centralized storage systems provide data durability and
availability of five (99.999\%) or more nines.

\section{Hypothesis}

In this work we want to answer the following questions:
\begin{enumerate}
    \item How effective is PoR in addressing the integrity problem in decentralized networks?
    \item Is a reputation system based on a ledger a viable measure against malicious nodes?
    \item How does the validation system affect the performance of the network?
    \item Does the validation system affect the other features of the system (performance, security, etc.)?
\end{enumerate}

\label{section:evaluation}
\section{Evaluation and testing}

Of the requirements, Scalability, Performance, Resilience, Reliability, Availability, and Security
are covered by other works.
We need to evaluate the Integrity requirement, which will be done by evaluating the verification/auditing mechanism,
the penalties and rewards for the nodes, and the reputation system.
Implementing the validation system could make the performance degrade, so we have to evaluate
if any of the requirements are affected by the validation system, and if so, how much.
We will discuss the details and the results of the evaluation in \ref{chapter:evaluation}.
