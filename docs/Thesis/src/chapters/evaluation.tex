\label{chapter:evaluation}
\chapter{Evaluation}

We need to evaluate the proper penalties and rewards for the nodes.
A good balance is needed between rewards and penalties to ensure that a node is
going to perform enough work for the system before being able to harm it.
Part of this evaluation will be:
\begin{enumerate}
    \item How much reputation points a node stakes when it stores a file?
    \item How much reputation points a node loses when it fails an audit?
    \item How much reputation points a node gains when it performs an audit?
    \item How much reputation points a node loses when it's discovered to not be performing audits?
    \item What is the starting reputation for new nodes?
    \item How do nodes increase their reputation at the very beginning?
    \item Does having high reputation give the node any benefits?
\end{enumerate}

Nodes with very high reputation could easily become malicious as they can absorb the penalties.
We need to solve this problem by either having a maximum reputation or by having the penalties be
percentage-based.

We need to answer the question of how often audits should be performed and how often nodes should
check the audit results of others.
The overhead of performing audits should be as low as possible, but the audits should be performed often enough
to ensure the integrity of the data.

Finding the balance between audits and the overhead of the audits, and the penalties and rewards for the nodes
is crucial for the success of the system.
We will evaluate the system by running it with different parameters and observing the behavior of the nodes.
We will discuss the details and the results of the evaluation in \ref{chapter:evaluation}.
