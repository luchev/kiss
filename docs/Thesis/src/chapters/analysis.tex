
% The one exception is storage attacks.
% No decentralized network has a solution for storage attacks.

% \subsection{Storage attacks}
% \label{section:storage-attacks}

% Storage attacks are the main focus of this thesis.
% They are mostly ignored by the literature, because decentralized networks are often designed to
% drop old files based on some criteria, such as popularity or age.
% This is done to save space and to keep the network up to date.
% However, if we want durable storage, we need to address these attacks.

% Storage attacks can be classified into two categories:
% \begin{enumerate}
%     \item \textbf{Data availability attacks} - an attacker claims to store data, but does not.
%     \item \textbf{Data integrity attacks} - an attacker claims to store data, but stores different data.
% \end{enumerate}

% Checking if a node stores the data it claims to store can be as simple as asking the node to return the data.
% However, this is a very inefficient and bandwidth-consuming method.
% Ideally, we would like to reduce the amount of traffic between nodes and still be able to check
% if the data is stored correctly.
