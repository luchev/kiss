\chapter{Introduction}
\label{chap:introduction}

Peer-to-peer (p2p) networks are distributed systems where there is no central authority.
A lot of the proposed networks implement a Distributed Hash Table (DHT), which is the basis for persistent file storage.

Such networks are considered resilient to attacks
because they are distributed and replicate the stored data.
However, some attacks can disrupt the integrity and
storage guarantees of a decentralized storage system.
Such attacks can be classified as:
\begin{itemize}
    \item Malicious peer deleting a stored file
    \item Malicious peer modifying a stored file
    \item Malicious peers overloading the system with requests (DDoS)
    \item Malicious peers joining the network in bulk, aiming to disrupt the established connections/topology
    \item A query dying in a malicious subnetwork
    \item Malicious peers performing a man-in-the-middle attack
    \item etc.
\end{itemize}

Most of these attacks have been addressed and there exist mechanisms to fend them off.
S/Kademlia\cite{skademlia} addresses the majority of such attacks, by building security mechanisms on top of Kademlia\cite{kademlia}.
In particular, S/Kademlia makes the following contributions:
\begin{itemize}
    \item Sybil Attack Prevention - preventing malicious peers from joining the network in bulk
    \item Secure Communication - encrypting the communication between peers
    \item Improved Node Authentication - using digital signatures to authenticate peers
    \item etc.
\end{itemize}

In this project, we are focusing on attacks that cannot be controlled via communication over the network, i.e., attacks that happen on the peers themselves — modifying/deleting stored data.
These attacks are possible because the peers in the network are usually not controlled by an authority, as is the case in centralized storage systems.
We attempt to prevent such attacks by continuously monitoring the stored files in the system.

Version 1 of the project focused on combining the decentralized storage from p2p systems with a centralized authority — a single Verifier node, which performed verification.
This centralized authority is a bottleneck to the scalability and goes against the idea of a decentralized network.
This Verifier kept a ledger of information on which peer (IP address) stored what file, and on a given period it would contact that peer and verify that the file was intact.

In Version 2, we are moving away from directly communicating with peers over IP and are instead moving that communication to it using the Kademlia network.
By doing this, the Verifier would now become just another peer.
This allowed us to merge the Verifier and Keeper nodes.
Now each Keeper runs a Verifier subprocess.
We discuss this in more detail in Chapter \ref{chap:implementation}.
We are still going to refer to the nodes with Keeper or Verifier, depending on which logical part we are referencing, but they are combined into one peer at the end.
To add to this, we are also distributing the responsibilities of the Verifiers, by making each one responsible only for a portion of the stored files.
This leaves the file catalog as the only centralized part of the system.

Having too many peers join the network rapidly might destabilize it in case many of these peers are malicious.
To avoid this, we are introducing a cryptographic puzzle, which must be solved before joining the network.
This restrains joining the network behind a time-consuming CPU procedure.
Such a procedure can stop Sybil attacks on the network.

Version 1 made use of simple hashing to verify the availability of files.
This relies on honest peers, that would rehash the file each time, and not just store the hashed value once and return it in subsequent queries.
This proof-of-concept procedure does not fare well with malicious peers in reality.
Therefore, in Version 2, we are switching it out for a Proof of Retrievability\cite{porfirst} protocol.
Proof of Retrievability protocols allows the client to store a small amount of metadata and perform many unique queries for the availability of a file on the server.
While the protocol is developed for a client-server architecture, it can be expanded to work in a peer-to-peer network if we imagine that the Verifiers are clients and the Keepers are servers.
Replacing simple hashing with a Proof of Retrievablity protocol allows us to detect malicious peers with higher probability.
