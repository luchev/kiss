\chapter{Related Work}
\label{chap:related-work}

\section{Proof of Retrievability}
Proof of Retrievability (PoR) is a fairly new concept introduced in 2007 Provable data possession at untrusted stores \cite{porveryfirtst}.
PoR is a cryptographic technique ensuring that a stored data object in a distributed system can be efficiently and verifiably retrieved, assuring data integrity and availability.
Later in 2009 “Proofs of retrievability: Theory and implementation” \cite{porfirst} provides a neat formal description of the field, that a lot of following papers use.
There are 2 state of the art PoR techniques:
\begin{itemize}
    \item From 2020, Dynamic proofs of retrievability with low server storage \cite{poralgebra} — based on an algebraic approach, involving matrix multiplication.
    \item From 2022, Efficient Dynamic Proof of Retrievability for Cold Storage \cite{pormerkle} — based on a purely cryptographic approach. This paper achieves the best bandwidth, client storage, and complexity overall, however, its implementation is incredibly complex.
\end{itemize}
While PoR is not exactly intended to be used in p2p systems, it fits the use case of this project nicely.

\section{Peer-to-peer networks}
Many decentralized storage systems have been proposed and implemented.
Modern distributed storage systems achieve up to logarithmic time for insert, lookup, and delete operations.
To name a few such systems — Chord \cite{chord}, Pastry \cite{pastry}, and Kademlia \cite{kademlia}.
The main difference between these systems is how they organize the network nodes.
Chord and Pastry use a ring structure, while Kademlia uses a binary tree.
Kademlia is more resilient to churn and performs faster lookups, but generates more network traffic \cite{kadvschordvspastry}.
Most of these systems serve as a foundation for applications that implement some decentralized storage,
but we won't go into detail about them, because they are not the focus of this project.

There are also multiple proposals on how to make decentralized storage systems more resilient to malicious peers.
To name a few — ARA \cite{ara}, Freenet \cite{freenet}, and S/Kademlia \cite{skademlia}.
The main idea behind these proposals is to have some kind of auditing mechanism that checks if the peers are behaving correctly.
ARA proposes a system where peers allocate “credits” to other peers they communicate with.
If peer A requests files from another peer B, B gains “credits”, while A informs other neighboring peers
that B has received “credits” from A.
Peers also share information about these “credits” with other peers and use them as an auditing mechanism.
S/Kademlia focuses on node authentication with expensive node ID generation and verifiable messages using
public and private key cryptography.
S/Kademlia also routes requests through multiple different nodes at the same time
to reduce the chance of a request falling into a malicious subnet.
Freenet's approach to security is to encrypt all data and route requests through multiple nodes.
It works in a way similar to Tor — nodes in the path to the target cannot tell what other nodes are in the path,
except for the previous and the next node.
Nodes storing the data are also obscured in the response, so the source cannot be tracked.
